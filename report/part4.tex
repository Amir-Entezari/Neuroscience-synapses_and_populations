
\section{فرایند تصمیم گیری}
    به بخش آخر پروژه، یعنی تصمیم گیری میرسیم. در این بخش از ما خواسته شده است که دو جمعیت نورونی تحریکی و یک جمعیت نورونی مهاری ساخته و ارتباطات بین آن ها را برقرار کنیم. سپس برای هر یک از جمعیت های نورونی تحریکی، یک جریان ورودی نویزی در نظر بگیریم و در هر بار شبیه سازی، یکی از ورودی ها به صورت کلی قوی تر از دیگری باشد.

    من ابتدا نتیجه خوبی که با پارامتر هایم گرفته ام را به عنوان نمونه میگذارم، سپس حالت های ممکن مختلف دیگر را بررسی میکنم.
    برای رسیدن به نمودار خواسته شده در این بخش، باید از دانشی که از بخش های قبل بدست آوردیم استفاده کنیم. به طور مثال اگر میخواهیم که پراکندگی زمان ضربه زدن نورون ها زیاد باشد، نیاز داریم 
    $j_0$ 
    را کاهش دهیم، یا اگر میخواهیم که تغییر رفتار جمعیت بعد از تصمیم گیری، مشهود باشد، نیاز داریم که وزن سیناپس بین جمعیت ها را بیشتر کنیم. من با درنظر گرفتن این موارد و تحلیل های قبلی، به شکل
    \ref{fig:part4-fixed-synapse}
    رسیدم که شبیه شکل آورده شده در پروژه بود. اما در ادامه، پارامتر ها را تغییر میدهیم تا تاثیر آن ها و روند رسیدن به این مدل را بهتر بفهمیم.
    \begin{figure}[!ht]
        \centering
        \includegraphics[width=0.9\textwidth]{plots/part4-fixed-synapse.pdf} 
        \caption{تصمیم گیری با دو جمعیت تحریکی و یک جمعیت مهاری}
        \label{fig:part4-fixed-synapse}
    \end{figure}
    همانطور که در شکل دیده می شود، در ابتدا جریان هر دو جمعیت نورونی تحریکی برابر است و از این رو جمعیت ها در تعادل قرار دارند و فعالیت هر سه جمعیت تقریبا یکسان است. اما بعد از اینکه در لحظه ۵۰۰ جریان جمعیت اول اندکی افزایش می یابد، در زمان کوتاهی، فعالیت آن و در نتیجه فعالیت جمعیت مهاری بیشتر شده و در نتیجه فعالیت جمعیت تحریکی دیگر کاهش می یابد. این به این دلیل است که با افزایش جریان ورودی به جمعیت ۱، و همچنین بیشتر شدن جریان سیناپس درون آن، نورون های آن سریع تر ضربه زده و در نتیجه فعالیت آن ها بیشتر می شود. با بیشتر شدن فعالیت جمعیت ۱، جریان ورودی به جمیعت مهاری نیز افزایش یافته و فعالیت آن نیز بیشتر میشود. اکنون جمعیت مهاری با بیشتر شدن فعالیتش میخواهد که فعالیت دو جمعیت دیگر را کاهش دهد، اما از انجا که جریان جمعیت تحریکی ۲ افزایشی نیافته بود، جریان ورودی به آن در مجموع کاهش یافته و فعالیت آن کم می شود.
    من برای آنکه نمودار بیشتر شبیه به نمودار پروژه شود، جریان های ورودی را نزدیک به جریان مورد نیاز برای ضربه زدن انتخاب کردم تا کاهش جریان بتواند تاثیر محسوسی در نمودار بگذارد. اما بیشتر بودن جریان نیز همانطور که از شکل
    \ref{fig:part4-fixed-synapse-high-curr}
    معلوم است تاثیر در این فرایند تصمیم گیری ندارد و صرفا فعالیت کلی را افزایش می دهد.
    (در ۲ مدل تحریکی، جریان مورد نیاز برای ضربه زدن، در حدود ۶ است)
    \begin{figure}[!ht]
        \centering
        \includegraphics[width=0.9\textwidth]{plots/part4-fixed-synapse-high-curr.pdf} 
        \caption{تصمیم گیری با دو جمعیت تحریکی با جریان ورودی بیشتر و یک جمعیت مهاری}
        \label{fig:part4-fixed-synapse-high-curr}
    \end{figure}

    حال تمامی پارامتر ها را برابر با یکدیگر قرار داده و تاثیر هر یک را روی فرایند تصمیم گیری بررسی میکنیم. نتیجه حاصل مطابق شکل 
    \ref{fig:part4-fixed-synapse-same-param} 
    خواهد شد.
    \begin{figure}[!ht]
        \centering
        \includegraphics[width=0.9\textwidth]{plots/part4-fixed-synapse-same-param.pdf} 
        \caption{تصمیم گیری با دو جمعیت تحریکی و یک جمعیت مهاری با پارامتر های یکسان}
        \label{fig:part4-fixed-synapse-same-param}
    \end{figure}

    از آنجا که واریانس تاثیر چشم گیری در فعالیت ندارد، از بررسی جداگانه آن چشم پوشی میکنیم.

    \paragraph*{پارامتر $j_0$}
    ابتدا بررسی میکنیم که اگر وزن های سیناپس های تحریکی به مهاری را افزایش دهیم چه اتفاقی می افتد. همانطور که در شکل 
    \ref{fig:part4-fixed-synapse-high-j-exc-inh}
    نیز مشاهده می شود، مانند حالت قبل فعالیت نورونی که دارای جریان بیشتر است افزایش می یابد اما به دلیل اینکه وزن ها به جمعیت مهاری نیز کمی بیشتر شده، نورون های مهاری نیز شروع به ضربه زدن میکنند و درنتیجه مقداری فعالیت جمعیت دارای جریان کمتر نیز کاهش می یابد.
    \begin{figure}[!ht]
        \centering
        \includegraphics[width=0.9\textwidth]{plots/part4-fixed-synapse-high-j-exc-inh.pdf} 
        \caption{تصمیم گیری با دو جمعیت تحریکی و یک جمعیت مهاری: افزایش $j_0$ در سیناپس های تحریکی به مهاری}
        \label{fig:part4-fixed-synapse-high-j-exc-inh}
    \end{figure}

    اگر همین کار را برای سیناپس های مهاری به تحریکی انجام میدادیم چه اتفاقی می افتاد؟ مطابق شکل
    \ref{fig:part4-fixed-synapse-high-j-inh-exc}
    مشاهده می شود که تغییر محسوسی در رفتار نسبت به شکل 
    \ref{fig:part4-fixed-synapse-same-param}
    رخ نمی دهد. چرا که افزایش وزن های سیناپس های مهاری به تحریکی، تنها در حالتی تاثیر گذار هست که وزن های تحریکی نیز زیاد باشند تا جمعیت مهاری فعالیتی داشته باشد.
    \begin{figure}[!ht]
        \centering
        \includegraphics[width=0.9\textwidth]{plots/part4-fixed-synapse-high-j-inh-exc.pdf} 
        \caption{تصمیم گیری با دو جمعیت تحریکی و یک جمعیت مهاری: افزایش $j_0$ در سیناپس های مهاری به تحریکی}
        \label{fig:part4-fixed-synapse-high-j-inh-exc}
    \end{figure}

    \paragraph*{پارامتر $n$}
    حال پارامتر 
    $n$ 
    را بررسی میکنیم. طبق چیزی که در بخش ۲ و ۳ یاد گرفتیم، میدانیم در جمعیت های یکسان، میتوان پارامتر 
    $p$ 
    را به گونه ای تنظیم کرد که تاثیری مشابه 
    $n$ 
    داشته باشد چرا که اثر هر دوی این پارامتر ها بر روی تعداد ارتباطات سیناپسی است، از این رو در این بخش فقط به بررسی 
    $n$ 
    میپردازیم.
    از این رو ابتدا تاثیر افزایش 
    $n$ 
    در سیناپس های تحریکی به مهاری را بررسی میکنیم. طبق شکل
    \ref{fig:part4-fixed-synapse-high-n-exc-inh.}
    میبینیم که تاثیر آن دقیقا مشابه بخش ۳ است و باعث می شود که فعالیت جمعیت مهاری کاهش یابد.
    (به همان علتی که در بخش ۳ گفته شد.) 
    تاثیر برعکس نیز هنگامی که فعالیت جمعیت مهاری زیاد باشد، مشابه است. در غیر این صورت تاثیر خاصی ندارد.
    \begin{figure}[!ht]
        \centering
        \includegraphics[width=0.9\textwidth]{plots/part4-fixed-synapse-high-n-exc-inh.pdf} 
        \caption{تصمیم گیری با دو جمعیت تحریکی و یک جمعیت مهاری: افزایش $n$ در سیناپس های تحریکی به مهاری}
        \label{fig:part4-fixed-synapse-high-n-exc-inh.}
    \end{figure}

    \paragraph*{دو الگوی دیگر}
        در نهایت، فرایند تصمیم گیری را با استفاده از دو الگوی دیگر انجام میدهیم. انتظار داریم که با وزن های ثابت مانند شکل 
        \ref{fig:part4-fixed-synapse}
        الگوی ارتباط کامل نتواند فرایند تصمیم گیری را به درستی انجام دهد، چرا که مانند آن است که در الگوی سوم، مقدار
        $n$ 
        را حداکثر بگیریم. اما در مورد الگوی دوم میتوانیم با انتخاب مناسب 
        $p$ 
        به رفتار مشابه به چیزی که در الگوی سوم دیدیم برسیم. شکل 
        \ref{fig:part4-full-synapse} و 
        شکل \ref{fig:part4-prob-synapse}
        این دو نتیجه گیری را تایید میکند.
        \begin{figure}[!ht]
            \centering
            \includegraphics[width=0.9\textwidth]{plots/part4-full-synapse.pdf} 
            \caption{تصمیم گیری با دو جمعیت تحریکی و یک جمعیت مهاری: الگوی اول}
            \label{fig:part4-full-synapse}
        \end{figure}
        \begin{figure}[!ht]
            \centering
            \includegraphics[width=0.9\textwidth]{plots/part4-prob-synapse.pdf} 
            \caption{تصمیم گیری با دو جمعیت تحریکی و یک جمعیت مهاری: الگوی دوم}
            \label{fig:part4-prob-synapse}
        \end{figure}
    \subsection{امتیازی}
        در بخش امتیازی از ما خواسته شده است که توزیع پتانسیل غشا را در طول فرایند شبیه سازی رسیم کنیم. من از دستیار حل تمرین مربوط درمورد این قسمت پرس و جو کردم و گفتند که این نمودار، باید به صورت هیت مپ 
        ($heat map$)
        باشد. از این رو، این نمودار برای فرایند تصمیم گیری با الگوی سیناپسی نوع سوم در شکل 
        \ref{fig:part4-fixed-synapse-bonus}
        آمده است.

        \begin{figure}[!ht]
            \centering
            \includegraphics[width=0.9\textwidth]{plots/part4-fixed-synapse-bonus.pdf} 
            \caption{تصمیم گیری با دو جمعیت تحریکی و یک جمعیت مهاری به همراه نمودار توزیع پتانسیل غشای جمعیت در طول فرایند}
            \label{fig:part4-fixed-synapse-bonus}
        \end{figure}