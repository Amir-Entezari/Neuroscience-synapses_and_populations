
\section{شبیه سازی دو جمعیت نورونی}
    \subsection{مورد الف}
    در این بخش، طبق خواسته پروژه با استفاده از یکی از مدل های نورونی که در پروژه قبل پیاده سازی کردیم، یکی از دو مدل گفته شده در فایل پروژه را انتخاب کرده و با استفاده از یک جریان ورودی تصادفی به عنوان ورودی نورون ها، فعالیت نورون ها را در گذر زمان بررسی میکنیم. سپس این آزمایش را برای مجموعه پارامتر های مختلف و الگو های ارتباطی متفاوت بین نورون ها انجام میدهیم و نتایج بدست آمده را تحلیل میکنیم. من از بین دو مدل گفته شده مدل اول را برای آزمایش انتخاب میکنم.

    در این مدل، دو جمعیت همگن از نورون ها را در نظر میگیریم که یکی شامل 
    $0.8 \times N$ 
    نورون تحریکی و دیگری شامل
    $0.2 \times N$ 
    نورون مهاری باشد.

    برای شروع، من ابتدا مدلی که پس از امتحان کردن پارامتر های متفاوت به آن رسیدم را بررسی میکنم، سپس حالت های مختلف با پارامتر های مختلف را آزمایش و تحلیل میکنم.
    این مدل، به طور کلی از ۱۰۰۰ نورون تشکیل شده که یک جمعیت آن دارای ۸۰۰ نورون تحریکی و جمعیت دیگر دارای ۲۰۰ نورون مهاری است. نورون های تحریکی از همان پارامتر های بخش قبل استفاده میکنند، اما نورون ها مهاری دارای کمی تغییر هستند. چرا که طبق گفته استاد در کلاس، نورون های تحریکی باید نرخ ضربه زدن سریع تری داشته باشند. از این رو، در این مدل، 
    $R$ 
    را تا ۱۰ افزایش داده و همچنین 
    $\tau$ 
    را تا ۳ کاهش میدهیم. برای سیناپس ها نیز، سه سیناپس در نظر میگیریم که یک سیناپس مربوط به سیناپس داخلی برای جمعیت تحریکی است و دو سیناپس دیگر به صورت رفت و برگشتی بین دو جمعیت هستند. همچنین هر سه سیناپس را نیز از نوع سوم، یعنی ارتباط تصادفی با تعداد ثابت نورون پیش سیناپسی انتخاب کردم. نکته دیگر قابل توجه این است که وزن هایی که از جمعیت تحریکی به جمعیت مهاری وجود دارد را نیز باید به نسبت بیشتر انتخاب کنیم، چرا که جمعیت مهاری جریان ورودی ندارد و جریان دریافتی آن را فقط جریان سیناپسی تشکیل میدهد. پس از شبیه سازی، نتیجه کلی مطابق شکل
    \ref{fig:part3-fixed-synapse-80-20-rand-curr-total}
    است. همچنین به عنوان اولین نمودار مثال، اختلاف پتانسیل را برای جمعیت تحریکی و مهاری نیز نمایش میدهیم ولی از آنجا که این نمودار اطلاعات زیادی دراختیار ما قرار نمیدهد از آوردن آن در ادامه اجتناب میکنیم و صرفا برای شهود بیشتر در اینجا می آوریم.
    (شکل \ref{fig:part3-fixed-synapse-80-20-rand-curr-exc}
    و شکل \ref{fig:part3-fixed-synapse-80-20-rand-curr-inh})
    \begin{figure}[!ht]
        \centering
        \includegraphics[width=0.9\textwidth]{plots/part3-fixed-synapse-80-20-rand-curr-exc.pdf} 
        \caption{رفتار جمعیت تحریکی وسیناپس نوع سوم به همراه جریان تصادفی}
        \label{fig:part3-fixed-synapse-80-20-rand-curr-exc}
    \end{figure}
    \begin{figure}[!ht]
        \centering
        \includegraphics[width=0.9\textwidth]{plots/part3-fixed-synapse-80-20-rand-curr-inh.pdf} 
        \caption{رفتار جمعیت مهاری وسیناپس نوع سوم به همراه جریان تصادفی}
        \label{fig:part3-fixed-synapse-80-20-rand-curr-inh}
    \end{figure}
    \begin{figure}[!ht]
        \centering
        \includegraphics[width=0.9\textwidth]{plots/part3-fixed-synapse-80-20-rand-curr-total.pdf} 
        \caption{رفتار جمعیت تحریکی و مهاری با نسبت ۸۰۰ به ۲۰۰ وسیناپس نوع سوم به همراه جریان تصادفی}
        \label{fig:part3-fixed-synapse-80-20-rand-curr-total}
    \end{figure}

    \paragraph*{تحلیل}
        همانطور که از شکل 
        \ref{fig:part3-fixed-synapse-80-20-rand-curr-total}
        نیز مشخص است، در ابتدا جمعیت تحریکی مقداری جریان دریافت می کند و اختلاف پتانسیل آن بالا میرود، پس از آنکه اولین ضربه ها توسط نورون های تحریکی زده می شود، مشاده میکنیم که جمعیت مهاری نیز جریان دریافت میکند و هنگامی که فعالیت جمعیت تحریکی زیاد می شود، جمعیت مهاری نیز از سیناپس های خود جریان دریافت کرده و فعالیت آن ها نیز زیاد می شود. زیاد شدن فعالیت جمعیت مهاری، منجر می شود جریان سیناپ منفی سی که از این جمعیت به جمعیت تحریکی وجود داردزیاد شود و فعالیت جمعیت تحریکی را کاهش دهد. از این رو مشاهده میکنیم که بعد از لحظاتی از هربار فعالیت زیاد جمعیت تحریکی، فعالیت جمعیت مهاری نیز زیاد شده و شاهد افت جریان زیاد در جمعیت تحریکی هستیم.
        (دره های نمودار جریان)
        و هر بار که فعالیت جمعیت تحریکی متوقف میشود، جریان ورودی به جمعیت مهاری نیز کم یا قطع شده که منجر می شود جمعیت تحریکی بتواند دوباره به فعالیت خود ادامه دهد و این روند تا پایان شبیه سازی ادامه یافته است.

        به طور خلاصه، پس از فعالیت جمعیت تحریکی شاهد افزایش فعالیت جمعیت مهاری، پس از افزایش فعالیت جمعیت مهاری شاهد کاهش جمعیت تحریکی و پس از کاهش فعالیت جمعیت تحریکی شاهد کاهش فعالیت مهاری هستیم.

        \paragraph*{دو مدل سیناپسی دیگر}
            به طور کلی رفتار دو مدل سیناپسی دیگر نیز مشابه مدل بررسی شده است و تفاوت در جزئیات آن ها است. برای مثال، با حفظ نسبت های پارامتر های سیناپس ها در شکل 
            \ref{fig:part3-fixed-synapse-80-20-rand-curr-total}
            میتوانیم این دو جمعیت را با سیناپس های دیگر نیز بسازیم و نتیجه مشابه را دریافت کنیم.
            (شکل \ref{fig:part3-full-synapse-80-20-rand-curr-total} 
            و شکل \ref{fig:part3-prob-synapse-80-20-rand-curr-total})
            \begin{figure}[!ht]
                \centering
                \includegraphics[width=0.9\textwidth]{plots/part3-full-synapse-80-20-rand-curr-total.pdf} 
                \caption{رفتار جمعیت تحریکی و مهاری با نسبت ۸۰۰ به ۲۰۰ وسیناپس نوع اول به همراه جریان تصادفی}
                \label{fig:part3-full-synapse-80-20-rand-curr-total}
            \end{figure}
            \begin{figure}[!ht]
                \centering
                \includegraphics[width=0.9\textwidth]{plots/part3-prob-synapse-80-20-rand-curr-total.pdf} 
                \caption{رفتار جمعیت تحریکی و مهاری با نسبت ۸۰۰ به ۲۰۰ وسیناپس نوع دوم به همراه جریان تصادفی}
                \label{fig:part3-prob-synapse-80-20-rand-curr-total}
            \end{figure}

            تفاوت هایی که مشاهده میکنیم، بیشتر درمورد میزان فعالیت های جمعیت ها است و رفتار کلی همان رفتاری است که بالاتر توضیح داده شد. 

        \subsection{ تحلیل پارامتر ها}
            حال سعی میکنیم تاثیر پارامتر های مختلف را روی رفتار این جمعیت مشاهده کنیم. کارمان را ابتدا روی همان الگوی سیناپسی سوم انجام میدهیم و اگر جایی نیاز به الگو های دیگر بود و رفتار جدیدی داشتند، آن ها را نیز بررسی میکنیم.

            \paragraph*{پارامتر $j_0$}
                در شکل 
                \ref{fig:part3-fixed-synapse-80-20-rand-curr-total}
                مشاهده کردیم که وزن های سیناپسی
                ($j_0$)
                برای ۳ سیناپس متفاوت بود. اگر این وزن ها را برابر قرار دهیم چه اتفاقی می افتد؟ جواب این سوال در شکل
                \ref{fig:part3-fixed-synapse-80-20-rand-curr-same-j}
                نهفته است. در این شکل، ما تمام پارامتر های موجود در سیناپس ها را یکسان گرفته ایم. در ادامه با بالا و پایین کردن مقدار 
                $j_0$ 
                تاثیر آن را روی رفتار جمعیت ملاحظه میکنیم.
                همانطور که در شکل 
                \ref{fig:part3-fixed-synapse-80-20-rand-curr-same-j}
                مشاهده میکنید، به علت وجود جریان تصادفی، زمان ضربه زدن نورون های مهاری با یکدیگر متفاوت است. از این رو فعالیت جمعیت مهاری نیز کم است. این فعالیت کم، بدین معناست که نورون هایی که به عنوان نورون پیش سیناپسی جمعیت مهاری هستند، با احتمال کمتری همزمان فعال شده و درنتیجه جریان ورودی به جمعیت مهاری به آستانه خود برای ضربه زدن نمی رسد. فقط در جاهایی از نمودار میبینیم که این نورون ها به صورت تک و توک ضربه زده اند.
                \begin{figure}[!ht]
                    \centering
                    \includegraphics[width=0.9\textwidth]{plots/part3-fixed-synapse-80-20-rand-curr-same-j.pdf} 
                    \caption{رفتار جمعیت تحریکی و مهاری با نسبت ۸۰۰ به ۲۰۰ وسیناپس نوع سوم به همراه جریان تصادفی و $j_0$ برابر}
                    \label{fig:part3-fixed-synapse-80-20-rand-curr-same-j}
                \end{figure}
                
                حال اگر مقدار 
                $j_0$ 
                در سیناپس تحریکی به مهاری را افزایش دهیم چه اتفاقی می افتد؟ شکل 
                \ref{fig:part3-fixed-synapse-80-20-rand-curr-increase-exc-inh-j}
                به ما میگوید که افزایش وزن های سیناپس تحریکی به مهاری، باعث می شود که جریان ورودی به جمعیت و نورون های مهاری بیشتری شده و در نتیجه، تعداد ضربات آن ها بیشتر شود. همانطور که از شکل نیز مشخص است، فعالیت نورون های مهاری بیشتر شده و این بیشتر شدن فعالیت باعث کاهش فعالیت نورون های تحریکی شده است.
                \begin{figure}[!ht]
                    \centering
                    \includegraphics[width=0.9\textwidth]{plots/part3-fixed-synapse-80-20-rand-curr-increase-exc-inh-j.pdf} 
                    \caption{رفتار جمعیت تحریکی و مهاری با نسبت ۸۰۰ به ۲۰۰ وسیناپس نوع سوم به همراه جریان تصادفی و $j_0$ بیشتر از تحریکی به مهاری}
                    \label{fig:part3-fixed-synapse-80-20-rand-curr-increase-exc-inh-j}
                \end{figure}

                حال اگر برعکس این اتفاق روی دهد چه؟ طبق شکل 
                \ref{fig:part3-fixed-synapse-80-20-rand-curr-increase-inh-exc-j}
                مشاهده میکنیم که این کار باعث می شود تا فعالیت نورون های جمعیت تحریکی مقداری کمتر شود که این کمتر شدن منجر می شود تا فعالیت خود نورون های مهاری نیز کاهش یابد و در کل، فعالیت کل جمعیت نیز کاهش یابد.

                \begin{figure}[!ht]
                    \centering
                    \includegraphics[width=0.9\textwidth]{plots/part3-fixed-synapse-80-20-rand-curr-increase-inh-exc-j.pdf} 
                    \caption{رفتار جمعیت تحریکی و مهاری با نسبت ۸۰۰ به ۲۰۰ وسیناپس نوع سوم به همراه جریان تصادفی و $j_0$ بیشتر از مهاری به تحریکی}
                    \label{fig:part3-fixed-synapse-80-20-rand-curr-increase-inh-exc-j}
                \end{figure}

                در نهایت بررسی میکنیم اگر مقدار 
                $j_0$ 
                در سیناپس بین نورون های جمعیت تحریکی افزایش یابد چه اتفاقی می افتد. از در بخش های قبلی میدانیم که افزایش وزن های سیناپس داخلی یک جمعیت باعث افزایش فعالیت جمعیت می شد. در شکل 
                \ref{fig:part3-fixed-synapse-80-20-rand-curr-increase-exc-exc-j}
                نیز این مورد را مشاهده میکنیم. اما این افزایش فعالیت، خود باعث می شود تا فعالیت جمعیت مهاری نیز افزایش یابد و این افزایش فعالیت نورون های مهاری، باعث کاهش فعالیت جمعیت تحریکی در بخش هایی میشود و از این رو قله و دره هایی در نمودار فعالیت جمعیت تحریکی و درنتیجه فعالیت کلی جمعیت ها ببینیم.
                \begin{figure}[!ht]
                    \centering
                    \includegraphics[width=0.9\textwidth]{plots/part3-fixed-synapse-80-20-rand-curr-increase-exc-exc-j.pdf} 
                    \caption{رفتار جمعیت تحریکی و مهاری با نسبت ۸۰۰ به ۲۰۰ وسیناپس نوع سوم به همراه جریان تصادفی و $j_0$ بیشتر از تحریکی به تحریکی}
                    \label{fig:part3-fixed-synapse-80-20-rand-curr-increase-exc-exc-j}
                \end{figure}

        \paragraph*{پارامتر $n$}
            حال به بررسی پارامتر 
            $n$ 
            می پردازیم. مانند پارامتر قبلی، ابتدا با زیاد کردن 
            $n$ 
            در سیناپس از جمعیت تحریکی به جمعیت مهاری شروع میکنیم. 
            همانطور که در شکل 
            \ref{fig:part3-fixed-synapse-80-20-rand-curr-increase-exc-inh-n}
            نیز مشاهده میکنیم، با افزایش 
            $n$ 
            از ۱۰ به ۲۰، فعالیت نورون های تحریکی کاهش می یابد.
            \begin{figure}[!ht]
                \centering
                \includegraphics[width=0.9\textwidth]{plots/part3-fixed-synapse-80-20-rand-curr-increase-exc-inh-n.pdf} 
                \caption{رفتار جمعیت تحریکی و مهاری با نسبت ۸۰۰ به ۲۰۰ وسیناپس نوع سوم به همراه جریان تصادفی و $n$ بیشتر از تحریکی به مهاری}
                \label{fig:part3-fixed-synapse-80-20-rand-curr-increase-exc-inh-n}
            \end{figure}
            
            در نگاه اول ممکن است انتظار داشته باشیم که افزایش 
            $n$ 
            منجر به افزایش فعالیت جمعیت پس سیناپسی شود ولی برعکس این اتفاق می افتد. دلیل این امر این است که ما می دانیم طبق فرمول 
            $w_{ij}$، 
            مقدار 
            $j_0$ 
            به تعداد 
            $n$ 
            نورون تقسیم می شود. به عنوان مثال اگر ما 
            $n$ 
            را ۱ درنظر بگیریم، هر نورون جمعیت پس سیناپسی فقط به یک نورون پیش سیناپسی متصل است و از آنجا که وزن آن نیز برابر با حداکثر مقدار، یعنی 
            $j_0$ 
            است، هر بار که نورون پیش سیناپسی مربوطه ضربه بزند، نورون پس سیناپسی نیز ضربه خواهد زد. اما اگر این مقدار را افزایش دهیم، مثلا ۲۰، 
            باعث می شود تا هر یک از نورون های پیش سیناپسی که متصل به نورون پس سیناپسی مورد نظر ما است، 
            وزنی حدود 
            $\frac{j_0}{20}$ 
            بگیرد که این باعث می شود تاثیر  تک نورون های پیش سیناپسی، کمتر شود یا به عبارت دیگر، دیگر ضربه زدن یک نورون پیش سیناپسی برای نورون پس سیناپسی مورد نظر ما کافی نباشد و باید تعداد بیشتری از بین این ۲۰ نورون همزمان ضربه بزنند.
        
            \paragraph*{پارامتر $p$}
                پارامتر 
                $p$ 
                نیز تاثیری مشابه پارامتر 
                $n$ 
                دارد. چرا که این پارامتر نیز تعیین کننده تعداد اتصالات بین نورون های دو جمعیت است. تنها تفاوتی که دارد، چون 
                $p$ 
                درصدی از جمعیت را انتخاب میکند، نمیتوان همانند الگوی سوم، با 
                $p$ 
                های برابر نتیجه مشابه شکل
                \ref{fig:part3-fixed-synapse-80-20-rand-curr-same-j}
                دریافت کرد، چرا که مثلا اگر 
                $p$ 
                را برابر 
                $0.1$ 
                قرار دهیم، تعداد اتصال های سیناپسی از جمعیت تحریکی به مهاری خیلی بیشتر از ۱۰ می شود! و ما مشاهده کردیم که با ۲۰ اتصال، فعالیت جمعیت مهاری بسیار کاهش میافت. به عنوان مثال برای رسیدن به نتیجه ای مشابه، میتوان پارامتر های سیناپس هارا به صورت شکل 
                \ref{fig:part3-prob-synapse-80-20-rand-curr-increase-exc-inh-p}
                قرار داد.
                \begin{figure}[!ht]
                    \centering
                    \includegraphics[width=0.9\textwidth]{plots/part3-prob-synapse-80-20-rand-curr-increase-exc-inh-p.pdf} 
                    \caption{رفتار جمعیت تحریکی و مهاری با نسبت ۸۰۰ به ۲۰۰ وسیناپس نوع دوم به همراه جریان تصادفی و $p$ متفاوت}
                    \label{fig:part3-prob-synapse-80-20-rand-curr-increase-exc-inh-p}
                \end{figure}

            \paragraph*{پارامتر اندازه جمعیت}
                در نهایت تاثیر اندازه جمعیت را روی رفتار بررسی میکنیم. تا کنون نسبت ۸۰۰ به ۲۰۰ را برای تحریکی به مهاری بررسی کردیم. اکنون می توانیم کمتری از آن، مثلا ۸۰ به ۲۰ را بررسی کنیم و البته پارامتر ها را ثابت بگیریم.
                (مثلا با پارامتر های شکل 
                \ref{fig:part3-fixed-synapse-80-20-rand-curr-same-j})
                همانطور که در شکل 
                \ref{fig:part3-fixed-synapse-80-20-rand-curr-less-size}
                ملاحظه میکنید، نگه داشتن پارامتر های ثابت با سیناپس نوع سوم، باعث می شود که به دلیل افزایش نسبی 
                $n$ 
                نسبت به اندازه جمعیت، فعالیت جمعیت به طور کلی کاهش بیابد که این، به دلیل افزایش نسبی
                $n$ 
                است که پیش تر توضیح داده شد.
                از این رو انتظار داریم که تغییر جمعیت در مدل سیناپسی دوم این مشکل را نداشته باشد.
                \begin{figure}[!ht]
                    \centering
                    \includegraphics[width=0.9\textwidth]{plots/part3-fixed-synapse-80-20-rand-curr-less-size.pdf} 
                    \caption{رفتار جمعیت تحریکی و مهاری با نسبت ۸۰ به ۲۰ وسیناپس نوع سوم به همراه جریان تصادفی}
                    \label{fig:part3-fixed-synapse-80-20-rand-curr-less-size}
                \end{figure}
